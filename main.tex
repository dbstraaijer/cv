%%%%%%%%%%%%%%%%%
% Dylan Straaijer - CV
% Based on AltaCV template version: v1.7.1b, 11 Jan 2024
%%%%%%%%%%%%%%%%

%% If you need to pass whatever options to xcolor
% \PassOptionsToPackage{dvipsnames}{xcolor}

\documentclass[10pt,a4paper,ragged2e,withhyper]{altacv}

%% Fork v1.6.5c: Overwriting sloppy environment to ignore any spaces and be used to solve overlapping cvtags
\newenvironment{sloppypar*}{\sloppy\ignorespaces}{\par}

% Change the page layout if you need to
\geometry{left=1.2cm,right=1.2cm,top=1cm,bottom=1cm,columnsep=0.75cm}

% The paracol package lets you typeset columns of text in parallel
\usepackage{paracol}

% Change the font if you want to, depending on whether
% you're using pdflatex or xelatex/lualatex
\ifxetexorluatex
  % If using xelatex or lualatex:
  \setmainfont{Roboto Slab}
  \setsansfont{Lato}
  \renewcommand{\familydefault}{\sfdefault}
\else
  % If using pdflatex:
  \usepackage[rm]{roboto}
  \usepackage[defaultsans]{lato}
  % \usepackage{sourcesanspro}
  \renewcommand{\familydefault}{\sfdefault}
\fi

\ifdarkmode%
  \definecolor{PrimaryColor}{HTML}{C69749}
  \definecolor{SecondaryColor}{HTML}{D49B54}
  \definecolor{ThirdColor}{HTML}{1877E8}
  \definecolor{BodyColor}{HTML}{ABABAB}
  \definecolor{EmphasisColor}{HTML}{ABABAB}
  \definecolor{BackgroundColor}{HTML}{191919}
\else%
  \definecolor{PrimaryColor}{HTML}{001F5A}
  \definecolor{SecondaryColor}{HTML}{0039AC}
  \definecolor{ThirdColor}{HTML}{F3890B}
  \definecolor{BodyColor}{HTML}{666666}
  \definecolor{EmphasisColor}{HTML}{2E2E2E}
  \definecolor{BackgroundColor}{HTML}{E2E2E2}
\fi%

\colorlet{name}{PrimaryColor}
\colorlet{tagline}{SecondaryColor}
\colorlet{heading}{PrimaryColor}
\colorlet{headingrule}{ThirdColor}
\colorlet{subheading}{SecondaryColor}
\colorlet{accent}{SecondaryColor}
\colorlet{emphasis}{EmphasisColor}
\colorlet{body}{BodyColor}
\pagecolor{BackgroundColor}

% Change some fonts, if necessary
\renewcommand{\namefont}{\Huge\rmfamily\bfseries}
\renewcommand{\personalinfofont}{\small\bfseries}
\renewcommand{\cvsectionfont}{\LARGE\rmfamily\bfseries}
\renewcommand{\cvsubsectionfont}{\large\bfseries}

% Change the bullets for itemize and rating marker
% for \cvskill if you want to
\renewcommand{\itemmarker}{{\small\textbullet}}
\renewcommand{\ratingmarker}{\faCircle}

%% sample.bib contains your publications
%% \addbibresource{main.bib}

\begin{document}
    \name{Dylan Straaijer}
    \tagline{Lead Developer \& Software Architect}
    %% You can add multiple photos on the left or right
    \photoL{4cm}{dylan}

    \personalinfo{
        \email{db.straaijer@gmail.com}\smallskip
        % \phone{+01-2345-678901}
        \location{Hoorn, Nederland}
        % \linkedin{linkedinUser}
        % \github{githubUser}
        %\homepage{nicolasomar.me}
    }
    
    \makecvheader
    %% Depending on your tastes, you may want to make fonts of itemize environments slightly smaller
    % \AtBeginEnvironment{itemize}{\small}
    
    %% Set columnratio (left small, right wide).
    \columnratio{0.25}

    % Start a 2-column paracol. Both the left and right columns will automatically
    % break across pages if things get too long.
    \begin{paracol}{2}
        % -----------------------------------------------------------
        % LINKER KOLOM
        % -----------------------------------------------------------
    
        % ----- TECH STACK -----
        \cvsection{Tech Stack}
        
            % Om de tags kleiner te maken, zetten we alles in \footnotesize
            {\footnotesize
            
            \textbf{Languages}\\
            \smallskip
            % Focus: Python & PHP voor backend, SQL voor data. Bash voor server.
            \cvtags{Python/true, PHP/true, Java, C++, SQL/true, JavaScript/true, Bash/true, C}
            \par\medskip
            
            \textbf{Cloud \& DevOps}\\
            \smallskip
            % Focus: AWS stack & CI/CD is key voor Lead rol.
            \cvtags{Lambda/true, Terraform/true, SQS/true, SNS/true, Docker/true, GitLab \& GitHub/true, Jenkins, IAM \& Identity Center/true, S3, DynamoDB/true, Heroku}
            \par\medskip
            
            \textbf{Frameworks \& Platforms}\\
            \smallskip
            % Focus: Je huidige werkervaring.
            \cvtags{Odoo/true, Mendix/true, Laravel/true, Bessy/true, Omnext/true, WordPress, Postman/true, Typer/true, Podio, Zapier}
            \par\medskip
            
            \textbf{Security \& Architectuur}\\
            \smallskip
            % Focus: Dit maakt jou senior.
            \cvtags{JWT/true, mTLS/true, ISO27001/true, Microservices/true, REST/true}
            \par\medskip
            
            \textbf{Academisch \& Overig}\\
            \smallskip
            % Wel noemen (laat intelligentie zien), niet highlighten.
            \cvtags{Assembly, LaTeX, Prolog, R, MatLab, Mathematica}
            
            } % Einde van footnotesize scope
        
        % ----- TECH STACK -----
        
        % % ----- LEARNING -----
        % \cvsection{Learning}
        %     \begin{sloppypar*}
        %         \cvtags{Uno, Dos/true, Tres, Cuatro/true, Cinco, Seis/true, Siete, Ocho/true, Nueve, Diez/true}
        %         \medskip

        %         \cvtags{Rojo/true, Amarillo, Azul/true, Verde, Violeta/true, Naranja, Marron/true, Blanco, Gris/true, Negro}
        %     \end{sloppypar*}
        % % ----- LEARNING -----
        
        % ----- LANGUAGES -----
        \cvsection{Talen}
            \cvlang{Nederlands}{Moedertaal}
            \medskip

            \cvlang{Engels}{Vloeiend / C1-C2}
            {\footnotesize (Tweetalig VWO \& Engelstalige studie)}
            \medskip

            \cvlang{Spaans}{Basis / A1}
        % ----- LANGUAGES -----
            
        % ----- REFERENCES -----
        % \cvsection{References}
        %     \cvref{Prof.\ Alpha Beta}{Institute}{a.beta@university.edu}
        %     \divider

        %     \cvref{Boss\ Gamma Delta}{Business}{g.delta@business.com}
        % ----- REFERENCES -----
        
        % ----- MOST PROUD -----
        % \cvsection{Most Proud of}
        
        % \cvachievement{\faTrophy}{Fantastic Achievement}{and some details about it}\\
        % \divider
        % \cvachievement{\faHeartbeat}{Another achievement}{more details about it of course}\\
        % \divider
        % \cvachievement{\faHeartbeat}{Another achievement}{more details about it of course}
        % ----- MOST PROUD -----
        
        % \cvsection{A Day of My Life}
        
        % Adapted from @Jake's answer from http://tex.stackexchange.com/a/82729/226
        % \wheelchart{outer radius}{inner radius}{
        % comma-separated list of value/text width/color/detail}
        % \wheelchart{1.5cm}{0.5cm}{%
        %   6/8em/accent!30/{Sleep,\\beautiful sleep},
        %   3/8em/accent!40/Hopeful novelist by night,
        %   8/8em/accent!60/Daytime job,
        %   2/10em/accent/Sports and relaxation,
        %   5/6em/accent!20/Spending time with family
        % }
        
        % use ONLY \newpage if you want to force a page break for
        % ONLY the current column
        \newpage
        
        %% Switch to the right column. This will now automatically move to the second
        %% page if the content is too long.
        \switchcolumn
        
        % ----- ABOUT ME -----
        \cvsection{Profiel}
            \begin{quote}
                Analytisch sterke en veelzijdige Lead Developer met een brede technische achtergrond. Ik ben geen 'one-trick pony', maar een generalist die complexe materie en nieuwe programmeertalen razendsnel doorgrondt. Met een academische achtergrond (Computer Science) en jarenlange ervaring in het opschalen van bedrijven, ben ik in staat om architectuur-beslissingen te nemen over de volledige breedte van de stack. Of het nu gaat om cloud-native oplossingen, legacy-migraties of security-implementaties: ik kies de tool die het beste past bij het probleem. Ik zoek een uitdagende omgeving waar technisch vernuft, robuustheid en veiligheid centraal staan.
            \end{quote}
        % ----- ABOUT ME -----
        
        % ----- EXPERIENCE -----
        \cvsection{Werkervaring}
            \cvevent{Lead Developer \& Software Architect}{Winst uit je woning}{Januari 2023 -- Heden}{Haarlem}
            \begin{itemize}
                \item Verantwoordelijk voor de technische visie, security en de transitie naar een event-driven architectuur.
                \item \textbf{Adaptievermogen:} In korte tijd diverse stacks (o.a. Odoo ERP, Mendix en AWS) eigen gemaakt om de nieuwe architectuur te leiden en verschillende systemen te integreren.
                \item \textbf{Security:} Lead in security-audits met grootbanken (ING). Implementatie van ISO27001 maatregelen, mTLS en Custom JWT flows.
                \item \textbf{Project GRIP:} Herontwerp van het core data model en implementatie van een message broker voor robuuste data-uitwisseling.
            \end{itemize}
            \divider
            
            \cvevent{IT Manager, Lead developer \& Security Officer}{Winst uit je woning}{Maart 2020 -- December 2022}{Haarlem}
            \begin{itemize}
                \item Opbouw en professionalisering van de interne IT-afdeling (gegroeid naar 7 werknemers). Introductie van CI/CD, Unit Testing en standaarden.
                \item \textbf{ISO27001:} Als Security Officer het ISMS opgezet en het bedrijf succesvol door de certificering geleid.
                \item Initiatiefnemer voor de transitie van legacy (Low-code/Podio) naar een volwassen Enterprise architectuur.
            \end{itemize}
            \divider

            \cvevent{Eigenaar / Full Stack Developer}{CRMDev (Freelance)}{Januari 2019 -- Maart 2020}{Hoorn}
            \begin{itemize}
                \item Greenfield development van een schaalbaar processysteem (PHP, SQL, API's).
                \item Verantwoordelijk voor full-stack: van Linux serverbeheer en hosting tot backend logica en frontend.
            \end{itemize}
            \divider

            \newpage

            \cvevent{Lead developer / Solution Architect}{CRM Fabriek}{2018 (6 maanden)}{Amsterdam}
            \begin{itemize}
                \item Technisch aansturen van een team van 4 developers. Vertalen van business requirements naar technische oplossingen.
            \end{itemize}
        % ----- EXPERIENCE -----
        
        % ----- EDUCATION -----
        \cvsection{Opleidingen}
            \cvevent{Bachelor Computer Science (Niet afgerond)}{Vrije Universiteit}{2017 -- 2019}{Amsterdam}
            \begin{itemize}
                \item 123 ECTS behaald, gemiddeld cijfer: \textbf{7,6}.
                \item Focus: Algoritmes, Datastructuren, Wiskunde (Java, C++, Python, Assembly).
                \item Studie gestaakt voor ondernemerschap; academisch denkniveau aantoonbaar.
            \end{itemize}
            \divider
            
            \cvevent{Bachelor Informatica (Propedeuse fase)}{Universiteit van Amsterdam}{2016 -- 2017}{Amsterdam}
            % \begin{itemize}
            %     \item GPA: 1,23
            % \end{itemize}
            \divider

            \cvevent{Tweetalig VWO (Natuur \& Gezondheid)}{Het Werenfridus}{2010 -- 2016}{Hoorn}
            % \begin{itemize}
                % \item 
            % \end{itemize}
        % ----- EDUCATION -----

        % ----- CERTIFICATES -----
        \cvsection{Certificaten}
            \cvevent{Mendix Developer Certification: Rapid Developer}{Mendix Academy}{2025}{}
            \begin{itemize}
                \item Behaald ter formalisering van jarenlange praktijkervaring met Mendix development en integraties.
            \end{itemize}
            \divider

            \cvevent{Informatiebeveiliging \& AVG}{Goodhabitz}{2024}{}
            \begin{itemize}
                \item Verdieping op het gebied van compliance en awareness in het kader van ISO27001.
            \end{itemize}
            \divider

            \cvevent{Cambridge English: Advanced (CAE)}{Cambridge English}{2014}{}
            \begin{itemize}
                \item Internationaal erkend certificaat op C1-niveau, behaald als onderdeel van tweetalig onderwijs.
            \end{itemize}
        % ----- CERTIFICATES -----
        
        % ----- PROJECTS -----
        \cvsection{Projects}
            \cvevent{Project 1 }{\cvreference{\faGithub}{https://github.com/user/repo}\cvreference{| \faGlobe}{https://project-demo.com/}}{Mm YYYY -- Mm YYYY}{}
            \begin{itemize}
                \item Item 1
                \item Item 2
            \end{itemize}
            \divider
            
            \cvevent{Project 2 }{\cvreference{\faGitlab}{https://gitlab.com/user/repo}\cvreference{| \faGlobe}{https://project-demo.com/}}{Mm YYYY -- Mm YYYY}{}
            \begin{itemize}
                \item Item 1
                \item Item 2
            \end{itemize}
        % ----- PROJECTS -----
    \end{paracol}
\end{document}